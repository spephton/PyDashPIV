\documentclass[12pt, a4paper]{amsart}
\usepackage{amsmath}
\usepackage{gensymb}
\usepackage{textcomp}
\usepackage[utf8]{inputenc}
\usepackage[a4paper, margin = 1in]{geometry}
\usepackage{bm}
\usepackage{graphicx}
\usepackage[hyphens]{url}


\title{A web-based python PIV analysis program for use in undergraduate laboratory experiments.}
\author{Jacob Sephton\\ supervised by Prof. Julio Soria}

\begin{document}
\maketitle

\section{Introduction and Aims}
\subsection{Introduction}
The measurement of velocity is one of the most fundamental parts of fluid mechanical analysis. Flows are characterised by their velocity fields, and from the velocity of a fluid other dynamic parameters, such as pressure, energy losses, and density, can be readily determined. Many methods exist for measuring velocity, but many commonly used methods, such as pitot tubes, restriction plates, or turbine meters only measure velocity at a point or as an average across a section, and may affect flow up and downstream of the measurement device \cite[pp. 109, 469]{munson}. These methods have the advantage of real-time measurement, but they cannot be used to determine the behaviour of a broader flow field within a flow: another method is required. 

Particle Image Velocimetry (henceforth: PIV) is one such method. In PIV, marker objects -- usually small spheres -- with neutral buoyancy are introduced to the portion of the flow to be studied \cite{wikipiv}\footnote{While not appropriate for a more in-depth analysis, Wikipedia provides reasonably good high level summaries of most concepts and is cited here for this reason.}. When appropriately sized for a specific flow, these particles will track the bulk motion -- the advection -- of the flow, with a high degree of accuracy. The velocity of the flow in the area around the particle -- both magnitude and direction -- can then be determined through photography over a known time interval. 

This makes PIV a powerful analytical tool that is commonly used when studying flow profiles [CITE ME]. When many particles are used, the velocity at many points in a flow field can be measured at once, permitting the interpolation of the entire velocity field in a given region of a flow. And, by varying the parameters of the study, like fluid properties, flow rate, particle size, and time interval, flow phenomena at different length and time scales can be modelled. In light of these potential benefits, undergraduate laboratory experiments to introduce the method represent an opportunity.

With opportunity comes challenges, however. Calculating the velocity at many points at once requires many particles be identified, and their displacements measured either in one frame or across two. 

\paragraph
what is particle image velocimetry
how is it accomplished
why is it useful
why is the development of this technology useful in this context -- undergraduate laboratory
what is Dash
why Dash?
what should the finished product do?

\subsection{Aims}
Specifically, this project aims to:
\begin{itemize}
\item develop a useful and user-friendly PIV application. This will involve: 
\begin{itemize}
	\item determining the specific requirements of the end user, deciding which features must be present and which features are simply nice-to-have, prioritising feature and product development.
	\item Determining the hardware and software operating environment \cite{worksafe}
	\item Determining the physical test equipment the software is to be used with \cite{wikipiv}
	\item Creating 
\end{itemize}
\end{itemize}

\bibliographystyle{ieeetr}
\bibliography{refs}

\end{document}