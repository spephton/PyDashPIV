\documentclass[12pt, a4paper]{amsart}
\usepackage{amsmath}
\usepackage{gensymb}
\usepackage{textcomp}
\usepackage[utf8]{inputenc}
\usepackage[a4paper, margin = 1in]{geometry}
\usepackage{bm}
\usepackage{graphicx}
\usepackage[hyphens]{url}


\title{A web-based python PIV analysis program for use in undergraduate laboratory experiments.}
\author{Jacob Sephton\\ supervised by Prof. Julio Soria}

\begin{document}
\maketitle

\section{Introduction and Aims}
\subsection{Introduction}
what is particle image velocimetry
how is it accomplished
why is it useful
why is the development of this technology useful in this context -- undergraduate laboratory
what is Dash
why Dash?
what should the finished product do?

\subsection{Aims}
Specifically, this project aims to:
\begin{itemize}
\item develop a useful and user-friendly PIV application. This will involve: 
\begin{itemize}
	\item determining the specific requirements of the end user, deciding which features must be present and which features are simply nice-to-have, prioritising feature and product development.
	\item Determining the hardware and software operating environment \cite{worksafe}
	\item Determining the physical test equipment the software is to be used with \cite{wikipiv}
	\item Creating 
\end{itemize}
\end{itemize}

\bibliographystyle{ieeetr}
\bibliography{refs}

\end{document}